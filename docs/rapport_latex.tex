\documentclass[12pt,a4paper]{article}

% Packages
\usepackage[utf8]{inputenc}
\usepackage[T1]{fontenc}
\usepackage{geometry}
\usepackage{graphicx}
\usepackage{float}
\usepackage{amsmath}
\usepackage{amssymb}
\usepackage{booktabs}
\usepackage{multirow}
\usepackage{array}
\usepackage{caption}
\usepackage{subcaption}
\usepackage{xcolor}
\usepackage{hyperref}
\usepackage{fancyhdr}

% Page setup
\geometry{left=2.5cm, right=2.5cm, top=2.5cm, bottom=2.5cm}
\setlength{\headheight}{15pt}
\setlength{\parindent}{0pt}
\setlength{\parskip}{6pt}

% Headers and footers
\pagestyle{fancy}
\fancyhf{}
\fancyhead[L]{\leftmark}
\fancyhead[R]{\thepage}
\renewcommand{\headrulewidth}{0.5pt}

% Hyperref setup
\hypersetup{
    colorlinks=true,
    linkcolor=blue,
    filecolor=magenta,      
    urlcolor=cyan,
    citecolor=blue
}

% Title page info
\title{\textbf{Analyse Multivariée du Dataset UCI Mushroom} \\ 
       \large ACM, Clustering et Analyse Discriminante}
\author{[Nom Personne A] \and [Nom Personne B] \and [Nom Personne C]}
\date{Janvier 2026 \\ 
      \vspace{0.3cm}
      \small Université [Nom] - Master [Spécialité]}

\begin{document}

% ==============================================================================
% PAGE DE GARDE
% ==============================================================================
\maketitle
\thispagestyle{empty}

\begin{abstract}
\noindent Ce rapport présente une analyse du dataset UCI Mushroom (8~124 champignons, 23 variables qualitatives) pour discriminer champignons comestibles et vénéneux. Nous appliquons une démarche structurée : Analyse des Correspondances Multiples (ACM), clustering sur composantes factorielles, et analyse discriminante. L'ACM révèle que les caractéristiques d'odeur et de surface constituent les axes principaux de variation (31,3\% d'inertie cumulée sur 5 axes).
\end{abstract}

\textbf{Mots-clés :} ACM, Classification non supervisée, Analyse discriminante, Données qualitatives

\newpage
\tableofcontents
\newpage

% ==============================================================================
% 1. INTRODUCTION
% ==============================================================================
\section{Introduction}

Le dataset UCI Mushroom regroupe 8~124 champignons décrits par 23 variables morphologiques qualitatives (forme du chapeau, odeur, couleur des lamelles, etc.). L'objectif est d'identifier les profils-types et les caractéristiques discriminantes pour la comestibilité.

\subsection{Démarche analytique}

Nous appliquons une méthodologie articulée en trois étapes :

\begin{enumerate}
    \item \textbf{ACM} : Réduction de dimensionnalité (23 variables $\rightarrow$ 5 axes factoriels, 31,3\% d'inertie)
    \item \textbf{Clustering} : Segmentation non supervisée sur composantes (CAH, K-means)
    \item \textbf{Analyse discriminante} : Modélisation prédictive edible/poisonous sur facteurs ACM
\end{enumerate}

Cette approche permet de combiner exploration et prédiction tout en valorisant la nature qualitative des données.

% ==============================================================================
% 2. DONNÉES
% ==============================================================================
\section{Données et préparation}

\subsection{Description du dataset}

\textbf{Source} : UCI Machine Learning Repository (\textit{Audubon Society Field Guide}, 1981)

\textbf{Dimensions} : $n = 8~124$ champignons, $p = 23$ variables qualitatives, $K = 111$ modalités totales

\textbf{Variable cible} : \texttt{class} $\in$ \{e (edible), p (poisonous)\}, distribution équilibrée (51,8\% vs. 48,2\%)

\begin{figure}[H]
\centering
\includegraphics[width=0.5\textwidth]{../reports/figures/desc_target_bar.png}
\caption{Distribution de la classe}
\label{fig:target_dist}
\end{figure}

\subsection{Preprocessing}

\textbf{Valeurs manquantes} : La variable \texttt{stalk-root} contient 2~480 valeurs "?" (30,5\%). Stratégie : imputation modale (modalité "b" = bulbous). Justification : préserve la distribution, évite la perte de 30\% des données, compatible ACM.

\subsection{Statistiques descriptives}

Le Tableau~\ref{tab:top_vars} présente les 6 variables clés.

\begin{table}[H]
\centering
\caption{Variables principales (top 6)}
\label{tab:top_vars}
\small
\begin{tabular}{lrrr}
\toprule
\textbf{Variable} & \textbf{$N_{mod}$} & \textbf{Top modalité} & \textbf{Freq. (\%)} \\
\midrule
class & 2 & e (edible) & 51,8 \\
odor & 9 & n (none) & 43,4 \\
gill-color & 12 & b (buff) & 21,3 \\
spore-print-color & 9 & w (white) & 29,4 \\
cap-color & 10 & n (brown) & 28,1 \\
gill-attachment & 2 & f (free) & 97,4 \\
\bottomrule
\end{tabular}
\end{table}

\textbf{Analyse bivariée odeur $\times$ classe} (Tableau~\ref{tab:odor_class}) : association quasi-parfaite. Les odeurs agréables (almond, anise) sont 100\% comestibles ; les odeurs fétides (foul, pungent) sont 100\% vénéneuses. Variable hautement discriminante.

\begin{table}[H]
\centering
\caption{Tableau croisé odeur $\times$ classe (extrait)}
\label{tab:odor_class}
\begin{tabular}{lrrr}
\toprule
\textbf{Odeur} & \textbf{Comestible} & \textbf{Vénéneux} & \textbf{Total} \\
\midrule
none (n)    & 3~408 & 120   & 3~528 \\
\textbf{foul (f)}    & \textbf{0}     & \textbf{2~160} & \textbf{2~160} \\
\textbf{almond (a)}  & \textbf{400}   & \textbf{0}     & \textbf{400}   \\
\textbf{anise (l)}   & \textbf{400}   & \textbf{0}     & \textbf{400}   \\
pungent (p) & 0     & 256   & 256   \\
\bottomrule
\end{tabular}
\end{table}

\begin{figure}[H]
\centering
\begin{subfigure}[b]{0.48\textwidth}
    \includegraphics[width=\textwidth]{../reports/figures/desc_top_modalities_odor.png}
    \caption{Odeur}
\end{subfigure}
\hfill
\begin{subfigure}[b]{0.48\textwidth}
    \includegraphics[width=\textwidth]{../reports/figures/desc_top_modalities_gill-color.png}
    \caption{Couleur lamelles}
\end{subfigure}
\caption{Distributions des modalités clés}
\label{fig:desc_modalities}
\end{figure}

% ==============================================================================
% 3. ACM
% ==============================================================================
\section{Analyse des Correspondances Multiples (ACM)}

\subsection{Méthodologie}

L'ACM transforme les 22 variables descriptives (111 modalités) en axes factoriels orthogonaux via le Tableau Disjonctif Complet (TDC). Inertie totale : $I_{tot} \approx 4.27$.

\subsection{Choix du nombre d'axes}

Nous conservons \textbf{k = 5 axes} (31,3\% d'inertie cumulée). Justification : coude visible après l'axe 5 (Fig.~\ref{fig:scree}), compromis interprétabilité/information.

\begin{table}[H]
\centering
\caption{Valeurs propres et inerties}
\label{tab:eigenvalues}
\begin{tabular}{lrrr}
\toprule
\textbf{Axe} & \textbf{$\lambda$} & \textbf{Inertie (\%)} & \textbf{Cumul (\%)} \\
\midrule
Dim1 & 0,324 & 7,59 & 7,59 \\
Dim2 & 0,295 & 6,91 & 14,49 \\
Dim3 & 0,271 & 6,33 & 20,83 \\
Dim4 & 0,243 & 5,68 & 26,51 \\
Dim5 & 0,203 & 4,76 & \textbf{31,27} \\
\bottomrule
\end{tabular}
\end{table}

\begin{figure}[H]
\centering
\includegraphics[width=\textwidth]{../reports/figures/acm_scree.png}
\caption{Scree plot et inertie cumulée}
\label{fig:scree}
\end{figure}

\subsection{Interprétation des axes factoriels}

\subsubsection{Axe 1 (7,59\%) : "Surface et Odeur"}

\textbf{Top contributions} (Table~\ref{tab:contrib_dim1}) : \texttt{ring-type\_\_l} (anneau large, 6,68\%), \texttt{stalk-surface-*\_\_k} (surface soyeuse, 6,4\%), \texttt{odor\_\_f} (odeur fétide, 5,49\%).

\begin{table}[H]
\centering
\caption{Top 5 contributions axe 1}
\label{tab:contrib_dim1}
\small
\begin{tabular}{lrr}
\toprule
\textbf{Modalité} & \textbf{Coord.} & \textbf{Contrib. (\%)} \\
\midrule
ring-type\_\_l (large)        & +1,73 & 6,68 \\
stalk-surface-below-ring\_\_k & +1,27 & 6,41 \\
odor\_\_f (foul)              & +1,21 & 5,49 \\
ring-type\_\_p (pendant)      & -0,67 & 3,05 \\
odor\_\_n (no odor)           & -0,62 & 2,36 \\
\bottomrule
\end{tabular}
\end{table}

\textbf{Interprétation} : Axe oppose champignons à texture lisse + odeur forte (pôle +, majoritairement vénéneux) vs. champignons sans odeur + anneau pendant (pôle -, neutres). Pouvoir discriminant fort.

\subsubsection{Axe 2 (6,91\%) : "Modalités rares"}

\textbf{Top contributions} : \texttt{gill-attachment\_\_a} (8,7\%, effectif 3\%), \texttt{stalk-color-*\_\_o} (7,2\%, effectif <1\%).

\textbf{Interprétation} : Effet de taille (modalités rares éloignées du barycentre). Oppose champignons atypiques vs. "moyens". Moins discriminant pour la classe, utile pour identifier sous-groupes.

\subsection{Visualisations et interprétation spatiale}

\begin{figure}[H]
\centering
\includegraphics[width=0.7\textwidth]{../reports/figures/acm_modalities_12.png}
\caption{Plan factoriel des modalités (axes 1-2)}
\label{fig:modalities}
\end{figure}

\textbf{Analyse spatiale des modalités} (Fig.~\ref{fig:modalities}) :
\begin{itemize}
    \item \textbf{Dispersion axe 1} : Opposition claire entre modalités à gauche (\texttt{odor\_\_n}, \texttt{ring-type\_\_p}) et à droite (\texttt{odor\_\_f}, \texttt{ring-type\_\_l}), confirmant l'interprétation "surface + odeur"
    \item \textbf{Modalités excentrées axe 2} : Les modalités rares (\texttt{gill-attachment\_\_a}, \texttt{gill-color\_\_y}) sont très éloignées du centre, illustrant l'effet de taille
    \item \textbf{Centre de gravité} : Les modalités fréquentes se concentrent autour de l'origine (\texttt{cap-shape\_\_x}, \texttt{gill-attachment\_\_f})
\end{itemize}

\begin{figure}[H]
\centering
\includegraphics[width=0.7\textwidth]{../reports/figures/acm_individuals_12_color_target.png}
\caption{Plan factoriel des individus (axes 1-2), colorés par classe}
\label{fig:individuals}
\end{figure}

\textbf{Analyse de la séparation des classes} (Fig.~\ref{fig:individuals}) :
\begin{itemize}
    \item \textbf{Tendance de séparation} : Les champignons vénéneux (rouge) se concentrent plutôt à droite de l'axe 1, les comestibles (vert) à gauche, cohérent avec les contributions (odor\_\_f vs. odor\_\_n)
    \item \textbf{Chevauchement important} : Les nuages se superposent fortement, indiquant que les axes 1-2 seuls (14,5\% inertie) ne suffisent pas pour une discrimination parfaite
    \item \textbf{Potentiel discriminant} : L'utilisation des 5 premiers axes (31,3\% inertie cumulée) dans l'analyse discriminante (Section~5) devrait significativement améliorer la séparation
\end{itemize}

\textbf{Conclusion ACM} : La réduction de dimensionnalité est effective (22 variables $\rightarrow$ 5 axes interprétables), avec un axe 1 montrant un fort pouvoir discriminant. La suite de l'analyse (clustering, discriminante) exploitera ces composantes pour modéliser la comestibilité.

\subsection{Export}

Fichiers générés : \texttt{mca\_coords.csv} (8~124 $\times$ 10 coordonnées), \texttt{mca\_eigenvalues.csv}, figures. Recommandation : utiliser k=5 axes pour clustering et analyse discriminante.

% ==============================================================================
% PLACEHOLDER SECTIONS 4-5 (PERSONNE B)
% ==============================================================================
\section{Clustering sur composantes ACM}
\textit{[Section rédigée par Personne B]}

\subsection{Méthode}
CAH et K-means sur coordonnées factorielles (k=5 axes).

\subsection{Résultats}
\textit{[Choix nombre de clusters, dendrogramme, profils]}

\section{Analyse discriminante}
\textit{[Section rédigée par Personne B]}

\subsection{Modèle}
LDA sur composantes ACM.

\subsection{Performance}
\textit{[Matrice de confusion, taux de succès, validation croisée]}

% ==============================================================================
% CONCLUSION
% ==============================================================================
\section{Conclusion}

L'ACM a révélé que les caractéristiques de surface et d'odeur constituent les axes principaux de variation (31,3\% d'inertie sur 5 axes). La variable \texttt{odor} présente une association quasi-parfaite avec la classe edible/poisonous, confirmée par l'axe 1. \textit{[À compléter avec résultats clustering et discriminante]}.

\textbf{Limites} : Inertie expliquée modérée (typique ACM), certaines modalités rares génèrent des effets de taille.

\textbf{Perspectives} : Comparer avec Random Forest, tester sur autres datasets mycologiques.



\end{document}
